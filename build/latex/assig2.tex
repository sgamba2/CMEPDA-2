%% Generated by Sphinx.
\def\sphinxdocclass{report}
\documentclass[letterpaper,10pt,english]{sphinxmanual}
\ifdefined\pdfpxdimen
   \let\sphinxpxdimen\pdfpxdimen\else\newdimen\sphinxpxdimen
\fi \sphinxpxdimen=.75bp\relax
\ifdefined\pdfimageresolution
    \pdfimageresolution= \numexpr \dimexpr1in\relax/\sphinxpxdimen\relax
\fi
%% let collapsible pdf bookmarks panel have high depth per default
\PassOptionsToPackage{bookmarksdepth=5}{hyperref}

\PassOptionsToPackage{warn}{textcomp}
\usepackage[utf8]{inputenc}
\ifdefined\DeclareUnicodeCharacter
% support both utf8 and utf8x syntaxes
  \ifdefined\DeclareUnicodeCharacterAsOptional
    \def\sphinxDUC#1{\DeclareUnicodeCharacter{"#1}}
  \else
    \let\sphinxDUC\DeclareUnicodeCharacter
  \fi
  \sphinxDUC{00A0}{\nobreakspace}
  \sphinxDUC{2500}{\sphinxunichar{2500}}
  \sphinxDUC{2502}{\sphinxunichar{2502}}
  \sphinxDUC{2514}{\sphinxunichar{2514}}
  \sphinxDUC{251C}{\sphinxunichar{251C}}
  \sphinxDUC{2572}{\textbackslash}
\fi
\usepackage{cmap}
\usepackage[T1]{fontenc}
\usepackage{amsmath,amssymb,amstext}
\usepackage{babel}



\usepackage{tgtermes}
\usepackage{tgheros}
\renewcommand{\ttdefault}{txtt}



\usepackage[Bjarne]{fncychap}
\usepackage{sphinx}

\fvset{fontsize=auto}
\usepackage{geometry}


% Include hyperref last.
\usepackage{hyperref}
% Fix anchor placement for figures with captions.
\usepackage{hypcap}% it must be loaded after hyperref.
% Set up styles of URL: it should be placed after hyperref.
\urlstyle{same}

\addto\captionsenglish{\renewcommand{\contentsname}{Contents:}}

\usepackage{sphinxmessages}
\setcounter{tocdepth}{1}



\title{Assig2}
\date{Oct 14, 2022}
\release{14/10/22}
\author{Sara Gamba}
\newcommand{\sphinxlogo}{\vbox{}}
\renewcommand{\releasename}{Release}
\makeindex
\begin{document}

\ifdefined\shorthandoff
  \ifnum\catcode`\=\string=\active\shorthandoff{=}\fi
  \ifnum\catcode`\"=\active\shorthandoff{"}\fi
\fi

\pagestyle{empty}
\sphinxmaketitle
\pagestyle{plain}
\sphinxtableofcontents
\pagestyle{normal}
\phantomsection\label{\detokenize{index::doc}}



\chapter{Indices and tables}
\label{\detokenize{index:indices-and-tables}}\begin{itemize}
\item {} 
\sphinxAtStartPar
\DUrole{xref,std,std-ref}{genindex}

\item {} 
\sphinxAtStartPar
\DUrole{xref,std,std-ref}{modindex}

\item {} 
\sphinxAtStartPar
\DUrole{xref,std,std-ref}{search}

\end{itemize}
\begin{description}
\sphinxlineitem{Module: basic Python}
\sphinxAtStartPar
Assignment \#4 (October 7, 2021)

\sphinxAtStartPar
— Goal
Create a ProbabilityDensityFunction class that is capable of throwing
preudo\sphinxhyphen{}random number with an arbitrary distribution.

\sphinxAtStartPar
(In practice, start with something easy, like a triangular distribution—the
initial debug will be easier if you know exactly what to expect.)

\sphinxAtStartPar
— Specifications
\sphinxhyphen{} the signature of the constructor should be \_\_init\_\_(self, x, y), where
\begin{quote}

\sphinxAtStartPar
x and y are two numpy arrays sampling the pdf on a grid of values, that
you will use to build a spline
\end{quote}
\begin{itemize}
\item {} 
\sphinxAtStartPar
{[}optional{]} add more arguments to the constructor to control the creation
of the spline (e.g., its order)

\item {} 
\sphinxAtStartPar
the class should be able to evaluate itself on a generic point or array of
points

\item {} 
\sphinxAtStartPar
the class should be able to calculate the probability for the random
variable to be included in a generic interval

\item {} 
\sphinxAtStartPar
the class should be able to throw random numbers according to the distribution
that it represents

\item {} 
\sphinxAtStartPar
{[}optional{]} how many random numbers do you have to throw to hit the
numerical inaccuracy of your generator?

\end{itemize}

\end{description}

\begin{sphinxVerbatim}[commandchars=\\\{\}]
\PYG{k+kn}{import} \PYG{n+nn}{numpy} \PYG{k}{as} \PYG{n+nn}{np}
\PYG{k+kn}{from} \PYG{n+nn}{scipy}\PYG{n+nn}{.}\PYG{n+nn}{interpolate} \PYG{k+kn}{import} \PYG{n}{InterpolatedUnivariateSpline}
\PYG{k+kn}{from} \PYG{n+nn}{matplotlib} \PYG{k+kn}{import} \PYG{n}{pyplot} \PYG{k}{as} \PYG{n}{plt}
\PYG{k+kn}{from} \PYG{n+nn}{scipy}\PYG{n+nn}{.}\PYG{n+nn}{optimize} \PYG{k+kn}{import} \PYG{n}{curve\PYGZus{}fit}



\PYG{k}{class} \PYG{n+nc}{ProbabilityDensityFunction}\PYG{p}{(}\PYG{n}{InterpolatedUnivariateSpline}\PYG{p}{)}\PYG{p}{:}

    \PYG{l+s+sd}{\PYGZdq{}\PYGZdq{}\PYGZdq{}Class describing a probability density function.}
\PYG{l+s+sd}{    Parameters}
\PYG{l+s+sd}{    \PYGZhy{}\PYGZhy{}\PYGZhy{}\PYGZhy{}\PYGZhy{}\PYGZhy{}\PYGZhy{}\PYGZhy{}\PYGZhy{}\PYGZhy{}}
\PYG{l+s+sd}{    x : array\PYGZhy{}like}
\PYG{l+s+sd}{        The array of x values to be passed to the pdf, assumed to be sorted.}
\PYG{l+s+sd}{    y : array\PYGZhy{}like}
\PYG{l+s+sd}{        The array of y values to be passed to the pdf.}
\PYG{l+s+sd}{    k : int}
\PYG{l+s+sd}{        The order of the splines to be created.}
\PYG{l+s+sd}{    \PYGZdq{}\PYGZdq{}\PYGZdq{}}

    \PYG{k}{def} \PYG{n+nf+fm}{\PYGZus{}\PYGZus{}init\PYGZus{}\PYGZus{}}\PYG{p}{(}\PYG{n+nb+bp}{self}\PYG{p}{,} \PYG{n}{x}\PYG{p}{,} \PYG{n}{y}\PYG{p}{,} \PYG{n}{k}\PYG{o}{=}\PYG{l+m+mi}{3}\PYG{p}{)}\PYG{p}{:}
        \PYG{l+s+sd}{\PYGZdq{}\PYGZdq{}\PYGZdq{}Constructor.}
\PYG{l+s+sd}{        \PYGZdq{}\PYGZdq{}\PYGZdq{}}
        \PYG{c+c1}{\PYGZsh{} Normalize the pdf, if it is not.}
        \PYG{n}{norm} \PYG{o}{=} \PYG{n}{InterpolatedUnivariateSpline}\PYG{p}{(}\PYG{n}{x}\PYG{p}{,} \PYG{n}{y}\PYG{p}{,} \PYG{n}{k}\PYG{o}{=}\PYG{n}{k}\PYG{p}{)}\PYG{o}{.}\PYG{n}{integral}\PYG{p}{(}\PYG{n}{x}\PYG{p}{[}\PYG{l+m+mi}{0}\PYG{p}{]}\PYG{p}{,} \PYG{n}{x}\PYG{p}{[}\PYG{o}{\PYGZhy{}}\PYG{l+m+mi}{1}\PYG{p}{]}\PYG{p}{)}
        \PYG{n}{y} \PYG{o}{/}\PYG{o}{=} \PYG{n}{norm}


        \PYG{n+nb}{super}\PYG{p}{(}\PYG{p}{)}\PYG{o}{.}\PYG{n+nf+fm}{\PYGZus{}\PYGZus{}init\PYGZus{}\PYGZus{}}\PYG{p}{(}\PYG{n}{x}\PYG{p}{,} \PYG{n}{y}\PYG{p}{,} \PYG{n}{k}\PYG{o}{=}\PYG{n}{k}\PYG{p}{)}\PYG{c+c1}{\PYGZsh{}super richiama i metodi di classi in altre classi.}
        \PYG{c+c1}{\PYGZsh{}inheritance=la classe prende i metodi/ gli attributi da altre classi}
        \PYG{c+c1}{\PYGZsh{} (da quella che è tra parentesi nel titolo della classe)}


        \PYG{n}{ycdf} \PYG{o}{=} \PYG{n}{np}\PYG{o}{.}\PYG{n}{array}\PYG{p}{(}\PYG{p}{[}\PYG{n+nb+bp}{self}\PYG{o}{.}\PYG{n}{integral}\PYG{p}{(}\PYG{n}{x}\PYG{p}{[}\PYG{l+m+mi}{0}\PYG{p}{]}\PYG{p}{,} \PYG{n}{xcdf}\PYG{p}{)} \PYG{k}{for} \PYG{n}{xcdf} \PYG{o+ow}{in} \PYG{n}{x}\PYG{p}{]}\PYG{p}{)}
        \PYG{c+c1}{\PYGZsh{}Return definite integral of the spline between two given points.}


        \PYG{n+nb+bp}{self}\PYG{o}{.}\PYG{n}{cdf} \PYG{o}{=} \PYG{n}{InterpolatedUnivariateSpline}\PYG{p}{(}\PYG{n}{x}\PYG{p}{,} \PYG{n}{ycdf}\PYG{p}{,} \PYG{n}{k}\PYG{o}{=}\PYG{n}{k}\PYG{p}{)}
        \PYG{c+c1}{\PYGZsh{}1\PYGZhy{}D interpolating spline for a given set of data points.}
        \PYG{c+c1}{\PYGZsh{}Fits a spline y = spl(x) of degree k to the provided x, y data.}
        \PYG{c+c1}{\PYGZsh{} Spline function passes through all provided points.}
        \PYG{c+c1}{\PYGZsh{} Equivalent to UnivariateSpline with s=0.}


        \PYG{c+c1}{\PYGZsh{} Need to make sure that the vector I am passing to the ppf spline as}
        \PYG{c+c1}{\PYGZsh{} the x values has no duplicates\PYGZhy{}\PYGZhy{}\PYGZhy{}and need to filter the y}
        \PYG{c+c1}{\PYGZsh{} accordingly:}
        \PYG{n}{xppf}\PYG{p}{,} \PYG{n}{ippf} \PYG{o}{=} \PYG{n}{np}\PYG{o}{.}\PYG{n}{unique}\PYG{p}{(}\PYG{n}{ycdf}\PYG{p}{,} \PYG{n}{return\PYGZus{}index}\PYG{o}{=}\PYG{k+kc}{True}\PYG{p}{)}
        \PYG{n}{yppf} \PYG{o}{=} \PYG{n}{x}\PYG{p}{[}\PYG{n}{ippf}\PYG{p}{]}
        \PYG{n+nb+bp}{self}\PYG{o}{.}\PYG{n}{ppf} \PYG{o}{=} \PYG{n}{InterpolatedUnivariateSpline}\PYG{p}{(}\PYG{n}{xppf}\PYG{p}{,} \PYG{n}{yppf}\PYG{p}{,} \PYG{n}{k}\PYG{o}{=}\PYG{n}{k}\PYG{p}{)}

    \PYG{k}{def} \PYG{n+nf}{prob}\PYG{p}{(}\PYG{n+nb+bp}{self}\PYG{p}{,} \PYG{n}{x1}\PYG{p}{,} \PYG{n}{x2}\PYG{p}{)}\PYG{p}{:}
        \PYG{l+s+sd}{\PYGZdq{}\PYGZdq{}\PYGZdq{}Return the probability for the random variable to be included}
\PYG{l+s+sd}{        between x1 and x2.}
\PYG{l+s+sd}{        Parameters}
\PYG{l+s+sd}{        \PYGZhy{}\PYGZhy{}\PYGZhy{}\PYGZhy{}\PYGZhy{}\PYGZhy{}\PYGZhy{}\PYGZhy{}\PYGZhy{}\PYGZhy{}}
\PYG{l+s+sd}{        x1: float or array\PYGZhy{}like}
\PYG{l+s+sd}{            The left bound for the integration.}
\PYG{l+s+sd}{        x2: float or array\PYGZhy{}like}
\PYG{l+s+sd}{            The right bound for the integration.}
\PYG{l+s+sd}{        \PYGZdq{}\PYGZdq{}\PYGZdq{}}
        \PYG{k}{return} \PYG{n+nb+bp}{self}\PYG{o}{.}\PYG{n}{cdf}\PYG{p}{(}\PYG{n}{x2}\PYG{p}{)} \PYG{o}{\PYGZhy{}} \PYG{n+nb+bp}{self}\PYG{o}{.}\PYG{n}{cdf}\PYG{p}{(}\PYG{n}{x1}\PYG{p}{)}

    \PYG{k}{def} \PYG{n+nf}{rnd}\PYG{p}{(}\PYG{n+nb+bp}{self}\PYG{p}{,} \PYG{n}{size}\PYG{o}{=}\PYG{l+m+mi}{1000}\PYG{p}{)}\PYG{p}{:}
        \PYG{l+s+sd}{\PYGZdq{}\PYGZdq{}\PYGZdq{}Return an array of random values from the pdf.}
\PYG{l+s+sd}{        Parameters}
\PYG{l+s+sd}{        \PYGZhy{}\PYGZhy{}\PYGZhy{}\PYGZhy{}\PYGZhy{}\PYGZhy{}\PYGZhy{}\PYGZhy{}\PYGZhy{}\PYGZhy{}}
\PYG{l+s+sd}{        size: int}
\PYG{l+s+sd}{            The number of random numbers to extract.}
\PYG{l+s+sd}{        \PYGZdq{}\PYGZdq{}\PYGZdq{}}
        \PYG{k}{return} \PYG{n+nb+bp}{self}\PYG{o}{.}\PYG{n}{ppf}\PYG{p}{(}\PYG{n}{np}\PYG{o}{.}\PYG{n}{random}\PYG{o}{.}\PYG{n}{uniform}\PYG{p}{(}\PYG{n}{size}\PYG{o}{=}\PYG{n}{size}\PYG{p}{)}\PYG{p}{)}
\end{sphinxVerbatim}



\renewcommand{\indexname}{Index}
\printindex
\end{document}